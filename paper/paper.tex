\documentclass[letterpaper,twocolumn,10pt]{article}
\usepackage{usenix2019_v3}

% to be able to draw some self-contained figs
\usepackage{tikz}
%\usepackage{parskip}
\usepackage{amsmath}
%\setlength{\parskip}{1pt} % 1ex plus 0.5ex minus 0.2ex}
\setlength\parskip{1pt}

\newcommand{\FigOverview}{
\begin{figure}[ht]
    \centering
    \includegraphics[width=0.9\linewidth,clip]{figures/dark-decoy-overview.png}
    \caption{\textbf{Dark Decoy Overview}\,---\, %
    }
    \label{fig:overview}
\end{figure}
}

\newcommand{\FigHighLevel}{
\begin{figure}[t]
    \centering
    \includegraphics[width=\linewidth,clip]{figures/high-level.png}
    \caption{\textbf{Overview}---%
      When deployed at an ISP, \scheme observes a mirror of inbound and outbound traffic.
      Users request service by embedding a steganographic registration message in a TLS handshake with any reachable site at the ISP\@. Then, the client connects to a random unused (``dark'')
      address in the ISP's AS, and \scheme communicates with the client as if it were that host.
    }
    \label{fig:overview}
\end{figure}
}


\newcommand{\yes}{\CIRCLE}
\newcommand{\no}{\Circle}
\newcommand{\maybe}{\LEFTcircle}

\newcommand{\TabCompare}{
\begin{table*}[ht]
    \centering
    \begin{tabular}{l|cccccccc}
            % Multiflow? Waterfall?
            & \rot{Telex~\cite{telex}} &
            \rot{Cirripede~\cite{cirripede}} &
            \rot{Decoy Routing~\cite{curveball}} &
            \rot{TapDance~\cite{tapdance}} & \rot{Rebound~\cite{rebound}} & \rot{Slitheen~\cite{slitheen}} & \rot{Waterfall~\cite{waterfall}} & \rot{\textbf{Dark Decoys}} \\
            \hline
                                      %Telex Cirr  DR     TD      RB    Slth   Water  DD
            No inline blocking        & \no & \no  & \no & \yes  & \no  & \no  & \no  & \yes \\
            Handles asym. routing     & \no & \yes & \no  & \yes & \yes & \no  & \yes & \yes \\
            Currently deployed        & \no & \no  & \no  & \yes & \no  & \no  & \no  & \no \\
            Replay attack resistant   &\yes & \yes & \yes & \no  & \yes & \yes & \yes & \yes \\
            Traffic analysis resitant &\no  & \no  & \no  & \no  &\maybe& \yes &\maybe& \no \\
            Uses unused addresses     & \no & \no  & \no  & \no  & \no  & \no  & \no  & \yes \\
    \end{tabular}
    %\caption{\textbf{Refraction Networking schemes}\,---\,}
    \label{tab:compare}
\end{table*}
}



\newcommand{\FigImplementation}{
\begin{figure}[ht]
    \centering
    \includegraphics[width=0.7\linewidth,clip]{figures/implementation.png}
    \caption{\textbf{Station Architecture}\,---\, %
    }
    \label{fig:implementation}
\end{figure}
}




\begin{document}

%don't print date
\date{}

% make title bold and 14 pt font (Latex default is non-bold, 16 pt)
\title{\Large \bf Dark Decoys:
  Conjuring Proxies from Unused Address Space}

%for single author (just remove % characters)
%\author{
%{\rm Your N.\ Here}\\
%Your Institution
%\and
%{\rm Second Name}\\
%Second Institution
%% copy the following lines to add more authors
%% \and
%% {\rm Name}\\
%%Name Institution
%} % end author

\maketitle

%-------------------------------------------------------------------------------
\begin{abstract}
%-------------------------------------------------------------------------------
Refraction Networking (formerly ``Decoy Routing'') has emrged as a useful tool in circumventing
        Internet censorship. By placing proxies in cooperating Internet Service
        Providers (ISPs) and using connections to existing reachable
        ``decoy'' sites for transport, censors cannot easily block access to
        such proxies without also blocking legitimate sites.

However, existing deployed Refraction Networking schemes such as TapDance suffer
from several problems, including a limited number of decoy sites in realistic
deployments, and an unfavorable tradeoff between performance and observability
by the censor. These problems limit where such proxies can be deloyed, hamper
their effectiveness, and may ultimately make them possible for censors to block.

In this paper, we present Dark Decoys, a deployable Refraction Networking scheme that
overcomes the limits on decoys and performance. Dark Decoys leverages the (primarily) unused address
space that is reachable through the deploying ISP. Rather than rely on existing
sites as reachable decoy sites that must be involved in each proxy connection,
Dark Decoys create seemingly legitimate hosts at new IP addresses. These
invented hosts are indistinguishable from normal hosts to the censor, but can
be used by clients as one-time proxies.

We implement our scheme ... and it's just the bee's knees. TODO

\end{abstract}


%-------------------------------------------------------------------------------
\section{Introduction}
%-------------------------------------------------------------------------------
% 1.5-2 pages

Censorship circumvention still important
-More countries blocking
-Existing countries blocking more

Existing circumvention strategies on shaky ground
-domain fronting going away
-active probing of existing proxies
-Cenosrs fingerprinting known protocols

Importance of Refraction Networking
-what it is, how it solves many of the above problems
-acknowledge challenge of ISP deployment, cite TapDance deployment as only refraction technology that has overcome this challenge so far

Challenges remain in Tapdance:
-Censor can fingerprint decoy sites
-Decoys themselves are limited (e.g. a few dozen in many cases)
-Can't have long-lived connections (performance and observability issue)
-Other performance limits (upload limit, TCP window size)

Dark decoys solve these issues
-Create new decoys from ``dark'' (unused) address space
-Clients register (via TapDance-like or other robust protocol (email, blockchain, whatever))
-Connect to custom IP address, talk whatever protocol client/station agrees on

One-time use address advantages:
-Censor cannot actively probe ahead of time (especially in IPv6), making it hard to fingerprint
-Decoys are now virtually unlimited (or limited only by address space)
-Connection can live as long as we want
-No pesky TCP/TapDance-y limits

Requirements/Challenges
-Censor can't be able to distinguish between dark decoy and legitimate hosts
    (otherwise they block all dark decoys)
    -Includes active probing: censor shouldn't be able to register existing IP
-Disruption avoidence: Want to pickup even for used addresses
    -Otherwise, censor probes to find truly unused address space and blocks
    -But can't disrupt legitimate services
    -Can overcome by limiting pickup to the registering client (but spoofing challenges...)
    -Note: also likely solved by the client-sends-SNI in the Mask site application...



\section{Background}
% 1 page
Background on TLS and Refraction Networking


\section{Threat Model}
% 0.5 page

-Censor can block arbitrary addresses (or networks), but faces a cost in doing so (collateral damage)
-Censor can know what network deploys the dark decoy station
-Censor knows the dark decoy prefixes distributed in the client, but they contain legitimate hosts
-Censor can use the client
-Censor can active probe limited sets, but cannot enumerate the entire prefix (i.e. IPv6 /32)
-Censor can active probe or (p)replay connections it suspects


\section{Architecture}
% 3 pages

Dark Decoys involve three main steps. First, clients \textbf{register} with the
ISP station,
and agree on the dark decoy IP address that will be used. Next, the client
connects to the agreed upon address, which is terminated by an
\textbf{application} running on the station. Finally, the client \textbf{proves}
to the application that it was the same entity that registered (and not a censor
attempting to probe), and is able to use the application as a proxy.

Similar to previous schemes, our design does not require expensive in-line
flow blocking, and can be accomplished with only a passive tap at the ISP.
Our architecture is also modular, in that the registration and application steps operate
independently, allowing a wide range of flexibility to evade censors. We
describe each of these components.

\subsection{Registration}

Clients can register intent to use the proxy in several covert ways. For
instance, they could use email or other existing intermitently available proxies
to register. In this section, we use a form of Refraction Networking similar to
TapDance to register directly with the station.

Registration is unidirectional (client to server) and does not require any
acknowledgement of receipt. To ensure receipt, clients can attempt to register
multiple times (possibly via multiple methods) with the same information, and expect
that one gets through.

\medskip

To register, clients connect to a \textbf{decoy} site that supports TLS. They
complete the handshake, and send a normal HTTPS request. In TLS, requests (and
responses) are encrypted and sent in an \texttt{Application Data} record. During
registration, the client alters the ciphertext of the \texttt{Application Data} record to
convey a short \textbf{tag} to the ISP station, on path between the client and
decoy site.

The tag contains a public key (encoded to be indistinguishable from random using
Elligator~\cite{elligator}), and a message encrypted under the shared secret
derived from a key exchange with the station's long-term public key hard-coded
in the client software. The station uses its private key to compute the same
shared secret from the (decoded) public key, and decrypts the message in the
tag. The censor, without knowledge of either the station or client's private
key, cannot derive the shared secret, preventing it from being able to decrypt
the message.


Inside the message, the client communicates a random \textbf{seed}, and other
configuration-specific information, such as flags to signify version, feature
support, and which set of dark decoy network prefixes the client knows of. The
client and server use a hash of the provided seed to determine which specific
dark decoy IP address to register. It may seem intuitive to instead have the client
send the specific IP address to register, but allowing the client arbitrary
choice also allows the censor to register suspected dark decoys and block them if
they pick up. By using a hash, the censor would have to pre-image the hash to
obtain a seed it could use to register for a desired IP address. It also gives a
secret (the seed) that only the client and station know, even after the client
connects to the dark decoy address.

Once the dark decoy IP address has been selected, the station watches for
packets destined to that address, and forwards them onto the dark decoy
\textbf{application}. The station can also optionally ignore packets not from
the source IP address of the registering client, so that to a probing censor,
the dark decoy appears to be firewalled off from all but the client.

%-IPv6 vs IPv4 dark decoys subsection?

\subsection{Address utilization}

% Note on IPv6 space? /32 of IPv6 is 96-bits to select...
Dark decoy IP addresses are selected from unused but still routed address
space that passes by the ISP. Traffic to unrouted addresses would produce ICMP
responses that a censor could observe, and are also easily blocked with knowledge
of a routing table. Because we support the use of IPv6 addresses, censors will
also be unable to scan ahead of time to determine if the host has been there
previously, or what was there before. While they can probe suspected dark decoy
addresses after clients connect to them, at that point the address has already been
registered, preventing the censor from identifying differences in behavior
before and after registration.

While address utilization in IPv4 is quite high (all blocks have been allocated
at the RIR level), the fraction of addresses that respond to TCP connections on
a given port is quite low: even popular ports like 443 (use by TLS) have less
than 2\% of IPv4 hosts respond to connection requests. This leaves a large
number of hosts available to use as dark decoys. However, if a client registers
a dark decoy address that is already happens to be used by an existing host,
there will be multiple responses when the client tries to connect, and the
connection will fail. However, this failure is still limited to the registering
client, as the application only responds to the IP of the client that
registered. In IPv6, the odds of picking a legitimate host are negligible: with
a /32 prefix of IPv6, there are $2^{96}$ addresses to choose from.


\subsection{Applications}


Once the client has registered, packets sent to the dark decoy IP address are
passed to the \textbf{application} running on the station. The application has
two main jobs: first, it must look like an innocuous legitimate service to the
censor, and second must allow the client to use it as a proxy.

Any protocol that satisfies these criteria can be used as an application. For
instance, \texttt{obfs4}~\cite{obfs4} is a pluggable transport used by
Tor~\cite{tor} that could be
modified to meet these needs: it requires each endpoint to have knowledge of a
shared secret (which we can derive from the seed shared by the client and
station during registration), and attempts to be a protocol with no discernible
fingerprint. Probing censors that do not have the shared secret receive no
response, while clients that have the secret can communicate with the proxy.

However, even though \texttt{obfs4} is indistinguishable from random, there are
still known attacks that can differentiate it from other
traffic~\cite{wang2015seeing}. While we have yet to see evidence of such attacks
employed in practice by censors, we consider two alternatives built on TLS.


\FigOverview

\subsubsection{Mask Sites}

TLS is a natural protocol for dark decoy applications, because it is ubiquitous on
the Internet (making it difficult for censors to block), while also providing
strong cryptographic protection against passive and active network adversaries.
However, there are several challenges to make it robust against censors that
wish to block a particular service.

This is because TLS sends important server-identifying content in plaintext
during the TLS handshake. This includes the Server Name Indication (SNI) in the
Client Hello message that sends the domain name of the server, and the
X.509 Certificate sent by the server.

To evade censors, we must send a plausible SNI value (sending no SNI is
uncommon and easily blocked---only 1\% of TLS connections do not carry the
SNI extension~\cite{tls-fingerprint}), and we must have the server respond with
a plausible (and corresponding) certificate. Even if we manage to avoid sending
either in the clear, censors could actively probe the server in a
way that would normally elicit a certificate.


We therefore attempt to mimic an existing \textbf{mask site}, by relaying
traffic between the client and a selected mask website. To a censor, this site
will be indistinguishable from the legitimate mask site, making it difficult for
them to block without potentially blocking the actual site. TLS connections to the
application will terminate exactly as connections to the mask site would, with the
application acting as a transparent proxy between the client and mask site.
However, this leaves the application unable to introspect on the contents of the
TLS connection to the mask site, as it does not have the client-site shared
secrets, and it cannot overtly man-in-the-middle the connection before knowing
it is communicating with the legitimate client (and not the censor).


To accomplish this, the client changes the shared secret it derives with the
mask site to something that the application can also derive. The client's first
\texttt{Application Data} packet is thus encrypted under a different secret than
the client/mask site secret. Specifically, the client uses the \textbf{seed}
sent during registration to derive the pre-master secret for the connection.
This is hashed along with the client and server randoms of the current (mask
site) TLS connection to obtain the master secret that determines
encryption/decryption/authentication keys.

The application can determine if the client did this by trial decryption with
the master secret derived from the known \textbf{seed}. If it succeeds, the
client has proved knowledge of the \textbf{seed}, and the application can
respond as a proxy. If not, the application simply continues to forward data
between the client and the mask site. As the censor does not have knowledge of
the \textbf{seed} used in registration, it cannot coerce the application to
appear as anything besides the mask site.


\paragraph{Mask Site Selection}

Selecting which sites to masquerade as must be done carefully to avoid censors
being able to detect obvious choices. For example, if a small university network
has a dark decoy in their network that appears to be \texttt{apple.com}, it
would be easy for a censor to block as a likely non-legitimate host. Likewise,
if a dark decoy at an IP address pretends to be a domain (\texttt{example.com}) that
globally resolves to a single different IP address, the censor could also
identify and block the dark decoy.

\smallskip
\noindent
\emph{Nearby sites} - One strategy for selecting mask sites to mimic is to pick websites that are
legitimately hosted in or near the network of the dark decoy addresses. This
effectively creates copies of legitimate sites, with the censor unable to
determine which copies are real and which are the dark decoys. However, as
mentioned, other signals such as DNS may reveal the true mask site.

\smallskip
\noindent
\emph{Alexa sites} -
An alternative strategy is to use popular sites, such as those from the Alexa
top million~\cite{alexa} list. As mentioned previously, it may be wise to avoid
sites that are obviously not hosted in the dark decoy address range, such as
large companies that run their own datacenters and own their own ASN.
The list could also be filtered to domains that resolve to different IP
addresses from different vantage points, making it harder for a censor to know
if a dark decoy corresponds to a domain's IP.

\smallskip
\noindent
\emph{Passive observation} -
The ISP tap could also collect sites by passively observing DNS requests, TLS
SNI, or certificates that pass by. This would allow for building a realistic set
of sites that are plausibly in the vicinity of the dark decoy addresses
that pass by the tap. It also has the added bonus of being able to discover
IPv6 addresses that could be used as decoys (for registration), where active
scanning would be infeasible.

In practice, clients can often try multiple dark decoys/mask sites over
several attempts, as blocking the client outright may negatively impact other
unrelated users behind the same network (e.g. in the case of NAT). Thus, even a
censor that can block most (but not all) mask site only delays access, and
doesn't prevent it outright.

\subsubsection{Encrypted SNI}

TLS~1.3~\cite{tls13} offers several features that may greatly simplify dark
decoy application design. For instance, TLS~1.3 handshakes include encrypted
certificates, potentially obviating the need to impersonate mask sites.
Unfortunately, TLS~1.3 currently still sends the SNI in the (plaintext) Client
Hello, meaning we would still have to choose a realistic domain to fool a
censor.

However, there are proposals to encrypt the SNI in the Client Hello~\cite{esni},
though none have been implemented or deployed as of early 2019. Nonetheless,
if widely adopted, Encrypted SNI (ESNI) would offer a powerful solution for dark
decoy applications by allowing the domain to remain hidden from the censor.
While the censor could still try to actively probe with guesses for the SNI,
servers could respond with generic ``Unknown SNI'' errors. If such responses
were common for incorrect SNI, the censor's efforts to identify dark decoys
would be frustrated.

\section{Implementation}
% 1 page

-ISP-scale implementation, able to handle 10+Gbps
-Language choice (Rust, Go, minimal C)
-Open source (client and station)

\section{Evaluation}
% 1 page

Compare to TapDance performance:
-Connection setup time
-Connection/session lifetime
-Throughput (especially upload vs download)
-Security considerations

\section{Attacks and Defenses}
% 1 page

-Fingerprinting Masked site vs Dark decoy application
-Active probing TLS connection
    -Block access unless from registering source IP
        -cite GFW taking over client IPs for active probing of old obfs
    -Verify knowledge of seed (but how to respond when verification fails?)

\section{Related Work}
% 0.5-0.75 page

MultiFlow, The Waterfall of Liberty, Slitheen, Rebound, TapDance, Cirripede, Telex, Decoy Routing
True Cost of RAD, Routing around Decoys
ISP-scale TapDance


\section{Future Work}
% 0.5 page

\section{Conclusion}
%.25 page

%%-------------------------------------------------------------------------------
%\section*{Availability}
%%-------------------------------------------------------------------------------
%
%USENIX program committees give extra points to submissions that are
%backed by artifacts that are publicly available. If you made your code
%or data available, it's worth mentioning this fact in a dedicated
%section.

%-------------------------------------------------------------------------------
\bibliographystyle{plain}
\bibliography{biblio}

\end{document}
