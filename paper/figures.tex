
\newcommand{\FigOverview}{
\begin{figure*}[t]
    \centering
    \includegraphics[width=0.60\linewidth,clip]{figures/dark-decoy-overview.png}
    \caption{\textbf{\scheme Overview}\,---\, %
    }
    \label{fig:overview}
\end{figure*}
}

\newcommand{\FigHighLevel}{
\begin{figure}[t]
    \centering
    \vspace{-0.5in}
    \includegraphics[width=\linewidth,clip]{figures/high-level.png}
    \caption{\textbf{Overview}---%
      \scheme is deployed at an ISP using a passive network tap.
      Clients request service by embedding a steganographic registration message in a TLS connection with any reachable site at the ISP\@. The client selects
      an unused (``dark'') address in the client's AS, and \scheme
      injects packets to communicate with the client as if it were a proxy server at that address.
    }
    \label{fig:highlevel}
\end{figure}
}


\newcommand{\FigIpBits}{
\begin{figure}[t]
    \centering
    \vspace{-0.5in}
    \includegraphics[width=\linewidth,clip]{figures/ip-bits-set-pdf.pdf}
    \caption{\textbf{IPv6 Entropy}---%
        We observed 30 minutes of IPv6 traffic on our tap to the /32 of the ISP's network.
        In each address observed, we counted the number of bits set, and compared this to
        randomly generated addresses. If the 96 non-network bits
        of each address were chosen randomly, we would expect to see a normal distribution
        (shown as Random). In practice, the distribution of observed IP addresses has much fewer
        bits set, suggesting that censors might be able to distinguish between dark decoy addresses
        and legitimate ones.
        %We note there are several /64 subnets that had many
        %addresses that appeared random (visibly overlapping the random distribution).
    }
    \label{fig:ipbits}
\end{figure}
}



\newcommand{\yes}{\CIRCLE}
\newcommand{\no}{\Circle}
\newcommand{\maybe}{\LEFTcircle}

\newcommand{\TabCompare}{
\begin{table*}[t]
    \centering
    \begin{tabular}{l|cccccccc}
            % Multiflow? Waterfall?
            & \rot{Telex~\cite{telex11}} &
            \rot{Cirripede~\cite{cirripede11}} &
            \rot{Decoy Routing~\cite{curveball11}} &
            \rot{TapDance~\cite{tapdance14}} & \rot{Rebound~\cite{rebound15}} & \rot{Slitheen~\cite{slitheen16}} & \rot{Waterfall~\cite{waterfall}} & \rot{\textbf{\scheme}} \\
            \hline
                                      %Telex Cirr  DR     TD      RB    Slth   Water  DD
            No inline blocking        & \no & \no  & \no & \yes  & \no  & \no  & \no  & \yes \\
            Handles asym. routing     & \no & \yes & \no  & \yes & \yes & \no  & \yes & \yes \\
            Currently deployed        & \no & \no  & \no  & \yes & \no  & \no  & \no  & \no \\
            Replay attack resistant   &\yes & \yes & \yes & \no  & \yes & \yes & \yes & \yes \\
            Traffic analysis resitant &\no  & \no  & \no  & \no  &\maybe& \yes &\maybe& \no \\
            Uses unused addresses     & \no & \no  & \no  & \no  & \no  & \no  & \no  & \yes \\
    \end{tabular}
    \caption{\textbf{Comparing Refraction Networking schemes}\,---\,}
    \label{tab:compare}
\end{table*}
}



\newcommand{\FigImplementation}{
\begin{figure}[ht]
    \centering
    \includegraphics[width=0.7\linewidth,clip]{figures/implementation.png}
    \caption{\textbf{Station Architecture}\,---\, %
    }
    \label{fig:implementation}
\end{figure}
}

\newcommand{\FigEvolution}{
\begin{figure}
  \centering
  \begin{subfigure}{\columnwidth}
    \includegraphics[width=\columnwidth]{figures/refraction-v1}
    \vspace{-3pt}
    
    \caption{\textbf{First generation systems} for Refraction Networking, such as Telex and Cirripede, operated as inline network elements, with the ability to observe traffic and block specific flows. ISPs worried that if the inline element failed, it could bring down the network.\looseness=-1}
    \label{fig:refraction-v1}
  \end{subfigure}
  \vspace{16pt}
  
  \begin{subfigure}{\columnwidth}
    \includegraphics[width=\columnwidth]{figures/tapdance}
    \vspace{-3pt}
    
    \caption{\textbf{TapDance} is a second-generation Refraction Network scheme that operates without flow blocking, needing only to passively observe traffic and inject packets. TapDance has recently been deployed at a mid-size ISP, but the techniques used to silence the decoy site and participate in the client--decoy TCP connection mid-stream add significant complexity, performance bottlenecks, and detection risk.}
    \label{fig:tapdance}
  \end{subfigure}
  \vspace{16pt}
  
  \begin{subfigure}{\columnwidth}
    \includegraphics[width=\columnwidth]{figures/dark-decoys}
    \vspace{-3pt}
    
    \caption{\textbf{\scheme}, our third-generation Refraction Networking design, overcomes these limitations.  It uses two sessions. First, the client connects to a decoy site and embeds a steganographic registration message, which the station receives using only a passive tap.  Second, the client connects to a ``dark address'' where there is no running server, and the station proxies the connection in its entirety.}
    \label{fig:dark-decoys}
  \end{subfigure}

  \vspace{12pt}
  \caption{\textbf{Evolution of Refraction Networking Approaches}}
\end{figure}
}
