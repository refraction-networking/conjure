\TabCompare

\section{Related Work}

We first compare \scheme with other Refraction Networking schemes and then discuss other related work.

\subsection{Prior Refraction Networking Schemes}

Since 2011, there have been several proposed Refraction Networking schemes.
Telex~\cite{telex11}, Cirripede~\cite{cirripede11} and Decoy
Routing (aka Curveball)~\cite{curveball11} are ``first generation'' protocols with nearly
identical features. These designs require inline flow blocking at the ISP to
allow the station to intercept flows with detected tags and act as the decoy
host for them. However, inline blocking is difficult for ISPs to deploy, as it
requires special-purpose hardware to be placed inline with production traffic,
introducing risk of failures and outages that may be expensive for the ISP
and potentially violate their contractual obligations (SLAs).

TapDance~\cite{tapdance14} solves the issue of inline-blocking by coercing the
decoy into staying silent, and allowing the station to respond instead. However,
as previously described, this trick comes at a cost: the decoy only stays silent
for a short timeout (typically 30-120 seconds), and limits the amount of data
the client can send before it responds. TapDance clients must keep connections
short and repeatedly reconnect to decoys, increasing overhead and potentially
alerting censors with this pattern. \scheme addresses this issue and allows
clients to maintain long-lived connections to the phantom host.

Rebound~\cite{rebound15} and Waterfall~\cite{waterfall17} both focus on routing
asymmetries and routing attacks by the censor. Rebound modifies the client's
packets on the way to the decoy, and uses error pages on the decoy site to
reflect data back to the client. Waterfall only observes and modifies the
decoy-to-client traffic, similarly using error pages on the decoy to reflect
communication from the client to the station. These schemes also provide some
resistance to traffic analysis, as they use the real decoy to reflect data to
the user. Thus, the TCP/TLS behavior seen by the censor more closely matches
that of a legitimate decoy connection. However, latency and other packet-timing
characteristics may be observable, and both schemes require some form of inline
flow blocking.

Slitheen~\cite{slitheen16} focuses on addressing observability by replacing
data in packets sent by the legitimate decoy. Thus, even the packet timings and
sizes of a Slitheen connection match that of a legitimate decoy connection.
However, Slitheen also requires inline-blocking, and introduces a large overhead
as it has to wait for the subset of data-carrying packets from the decoy that
Slitheen can safely replace. We note that the Slitheen model of mimicry is
compatible with \scheme, as we could use Slitheen as the application
protocol. Despite using a passive tap, our scheme is effectively inline to the
phantom host (which won't otherwise respond).


Bocovich and Goldberg propose an asymmetric gossip scheme~\cite{secureasymmetry} 
that combines a passive monitor on the forward path from the client to the decoy 
with an inline blocking element on the return path. These elements work in concert 
to allow schemes such as Telex and Slitheen to work on asymmetric connections. 
This approach, however, still requires inline blocking on one direction, and 
further complicates deployment by requiring the installation of more components 
and potentially complex coordination between them. MultiFlow~\cite{multiflow} 
uses refraction networking only as a forward mechanism to communicate a web 
request to the station, and then uses a bulletin board or email to deliver the 
response back. It does not require inline flow blocking as it does not modify 
users' traffic at all, but it fundamentally relies on a separate data delivery 
mechanism, similar to other cloud- or email-based circumvention 
tools~\cite{SWEET-ToN,CloudTransport}.

\scheme allows a large amount of flexibility compared to previous schemes.
Because we have significant degrees of freedom in choosing the specific
application the phantom host will mimic or talk, our scheme can combine the best
of existing Refraction Networking protocols to achieve high performance, be easy
to deploy, and
also be resistant to active attacks such as replaying or probing by the censor.
Table~\ref{tab:compare} lists the existing Refraction Networking schemes and
their features, as compared to \scheme.

\subsection{Decoy Placement and Routing Attacks}

Houmansadr et al.~\cite{cirripede11} found that placing refraction proxies in a 
handful of Tier 1 networks would be sufficient for them to be usable by the 
majority of the Internet population. Cesareo et al.~\cite{decoy-placement} 
developed an algorithm for optimizing the placement of proxies based on AS-level 
Internet topology data. Schuchard et al.~\cite{rad12} suggested that a censor may 
actively change its routes to ensure traffic leaving its country avoids the 
proxies, but Houmansadr et al.~\cite{true-cost-rad} suggested that real-world 
constraints on routing make this attack difficult to carry out in practice. 
Nevertheless, Nasr et al.~\cite{game-of-decoys} propose a game-theoretic 
framework to optimize proxy placement in an adversarial setting, and the 
design of Waterfall~\cite{waterfall17} is in part motivated by resilience to 
routing attacks, as it is more difficult for the censor to control the return 
path from a decoy site, rather than the forward path.

In practice, deployment of refraction networking has so far been at access, 
rather than transit ISPs~\cite{frolov2017isp}. This may be in part because a 
transit ISP has a large number of routers and points-of-presence, significantly 
raising the costs of deployment~\cite{devil-details}.\footnote{We
note that Gosain et al.~\cite{devil-details} use an estimate of \$885,000/proxy, 
while Frolov et al.~\cite{frolov2017isp} report line-rate TapDance deployment 
using commodity hardware that costs only several thousand dollars.} Likewise, 
we expect \scheme to use address space announced by the ISP, rather than 
addresses relayed by it, which mitigates routing-based attacks. Depending on 
the size of the ISP, however, a censor may decide to block the entirety of its 
address space, which would incur smaller collateral damage than blocking all 
addresses seen by a transit ISP.

\subsection{Avoiding Destination Blocking}

Traditionally, proxies deployed for censorship are eventually identified and
blocked by the censor. Several proposals have been made to carefully control
the distribution of proxy addresses, using social connections and
reputation~\cite{proximax,rbridge,salmon}. Nevertheless, keeping this
information secret is challenging; additionally, censors often employ active
scanning techniques to discover proxies~\cite{Dunna2018a}. Refraction
networking generally assumes that clients have no secret information, and
instead relies on the  collateral damage that would result from blocking all
the potential decoy destinations. \scheme furthers this goal by creating a
large number of destinations out of the dark space. A similar approach was
conceptualized in DEFIANCE~\cite{lincoln2012}, where censored Tor clients
connect to pools of addresses that are volunteered to run Tor bridge nodes.
Unlike \scheme, DEFIANCE requires network admins that own the IP blocks to
participate and operate proxies on their addresses, allowing censors to block
the IP ranges involved once discovered. In contrast, \scheme can be
deployed at transit ISPs, and censors face a large risk of collateral
over-blocking if they block the network blocks used.
Another similar approach was
taken by CensorSpoofer~\cite{censorspoofer}, which spoofed traffic from a large
set of dummy destinations. CensorSpoofer, however, could only send information
in one direction---to the client---and had to rely on a separate out-of-band
channel for client-to-proxy communication. As an alternative approach,
FlashProxy~\cite{flash-proxies} and Snowflake~\cite{snowflake} allow users to
run Flash- or WebRTC-based proxies within their browser to allow censored users
to connect to the Tor network with the potential to greatly increase the
number. In practice, these proxies served a very small number of users, as
compared with other Tor bridge transports.\footnote{\url{https://metrics.torproject.org/userstats-bridge-transport.html?start=2017-01-01&end=2019-02-15&transport=!<OR>&transport=websocket&transport=snowflake}}
